\documentclass[12pt,a4paper]{report}
\usepackage[utf8]{inputenc}
\usepackage{amsmath}
\usepackage{amsfonts}
\usepackage{amssymb}
\input defs.tex
\bibliographystyle{alpha}


\title{Neural Network Based Decoding over a Molecular Communication Channel}
\author{Peter Hartig}

\begin{document}
\maketitle

\begin{abstract}
The general communication channel is equivalent to a conditional probability $P(x|y)$. This takes into account the transmitted information, the (potentially random) channel through which this information passes as well as random noise added to the information prior to being received. The communication problems optimizes a form of $P(x|y)$ over a set of transmitted information. In general such problems do not require perfect knowledge of the distribution $P(x|y)$. Such solutions are attractive when $P(x|y)$ is unknown or difficult to obtain. In this work, forms of $P(x|y)$ are used so as to allow for the use of a neural network in finding a solution to the communication problem. 
\end{abstract}

\newpage
\tableofcontents
\newpage

\section{Introduction}


\section{System Model}
\subsection{Maximum Likelihood}
\subsection{Maximum A Posteriori}


\section{Results}

\section{Conclusion}

\newpage
\bibliography{mc_report}
\end{document}